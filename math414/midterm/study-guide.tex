\documentclass[11pt]{article} % 

\setlength{\oddsidemargin}{-0.15in}
\setlength{\topmargin}{-0.5in}
\setlength{\textwidth}{6.5in}
\setlength{\textheight}{9in}
\setlength\parindent{0pt}

\newcommand{\cP}{{\cal P}}
\newcommand{\IN}{{\bf N}}
\newcommand{\IZ}{{\bf Z}}
\newcommand{\IR}{{\bf R}}
\newcommand{\IQ}{{\bf Q}}
\newcommand{\points}[1]{{\it (#1 Points)}}
\newcommand{\qed}{{\hfill {\rm QED}}}
\newcommand\tab[1][1cm]{\hspace*{#1}}


\begin{document} 

{\bf \underline{Special Matrices}}


\medskip
{\bf Diagonally Dominant}

A matrix is {\bf diagonally dominant} when

\begin{center}
$|a_{ii}| \geq \sum_{j=1,j \neq i}^{n} |a_{ij}|$ holds for each $i = 1,2,...,n$.
\end{center}

That is, the absolute value of the element on the diagonal is greater than the sum of the absolute values of all other elements in the row.

It is {\bf strictly diagonally dominant} when,

$$|a_{ii}| > \sum_{j=1,j \neq i}^{n} |a_{ij}|$$


\vskip .4in
{\bf Symmetric Positive Definite Matrices (SPD)}

A matrix $A$ is {\bf symmertric positive definite} if it is symmetric and if $x^tAx > 0$ for every $n$-dimensional vector $x \neq 0$.

If $A$ is an $n \times n$ SPD matrix, then

\begin{itemize}
	\item $A$ has an inverse
	\item $a_{ii} > 0$ for each $i = 1, 2, ..., n$
	\item $max_{1 \leq k, j \leq n} |a_{kj}| \leq max_{1 \leq i \leq n} |a_{ii}|$
	\item $(a_{ij})^2 < a_{ii}a_{jj}$, for each $i \leq j$
\end{itemize}

The matrix $A$ is positive definite if and only if $A$ can be factored in the form $LDL^t$, where $L$ is lower triangular with $1$s on its diagonal and $D$ is a diagonal matrix with positive diagonal entries.

The matrix $A$ is positive definite if and only if $A$ can be factored in the form $LL^t$, where $L$ is lower triangular with nonzero diagonal entries.

\vskip .4in
{\bf Cholesky ($LL^T$)}


\vskip .4in
{\bf Tridiagonal}

$O(n)$ time multiplication


\vskip .4in
{\bf Vector Norms}

A vector norm on $R^n$ is a function, $|| \cdot ||$, from $R^n$ into $R$ with the following properties:

\begin{itemize}
	\item $||x|| \geq 0$ for all $x \in R^n$
	\item $||x|| = 0$ if and only if $x = 0$
	\item $||\alpha x|| = |\alpha| ||x||$ for all $\alpha \in R$, $x \in R^n$
	\item $||x + y|| \leq ||x|| + ||y||$ for all $x, y \in R^n$
\end{itemize}

The $l_2$ norm is defined as such:

$$||x||_2 =  \left\{ \sum_{i=1}^{n} x_{i}^{2} \right\}^{\frac{1}{2}}$$

The $l_{\infty}$ norm is defined as such:

$$||x||_{\infty} =  max_{1 \leq i \leq n} |x_{i}|$$

The distance between two vectors is defined as the norm of the difference of the vectors.

\medskip
A sequence $\{x^{(k)}\}_{k=1}^{\infty}$ of vectors in $R^n$ is said to converge to $x$ with respect to the norm $|| \cdot ||$ if, given any $\epsilon > 0$, there exists an integer $N(\epsilon)$ such that

\begin{center}
$|| x^{(\epsilon)} - x || < \epsilon$, for all $k \geq N(\epsilon)$
\end{center}

\medskip
The sequence of vectors $\{x^{(k)}\}$ converges to $x$ in $R^n$ with respect to the $l_{\infty}$ norm if and only if $lim_{k \rightarrow \infty} x^{(k)} = x_i$, for each $i = 1,2,...,n$.



\vskip .4in
{\bf Matrix Norms}

A matrix norm on the set of all $n \times n$ matrices is a real-valued function, $|| \cdot ||$, defined on this set, satisfying for all $n \times n$ matrices $A$ and $B$ and all real numbers $\alpha$:

\begin{itemize}
	\item $||A|| \geq 0$
	\item $||A|| = 0$ if and only if $A$ is $0$, the matrix with all $0$ entries
	\item $|| \alpha A || = |\alpha| ||A||$
	\item $|| A + B || \leq ||A|| + ||B||$
	\item $||AB|| \leq ||A|| ||B||$
\end{itemize}

The distance between $n \times n$ matrices $A$ and $B$ with respect to this matrix norm is $||A − B||$.

If $|| \cdot ||$ is a vector norm on $R^n$, then

$$||A|| = max_{||x||=1} ||Ax||$$

is a matrix norm.

\medskip
For any vector $z \neq 0$, matrix $A$, and any natural norm $|| \cdot ||$, we have

$$||Az|| \leq ||A|| \cdot ||z||$$


\medskip
If $A = (aij)$ is an $n \times n$ matrix, then
$$||A||_{\infty} = max_{1 \leq i \leq n} \sum_{j=1}^{n} |a_{ij}|$$
(max row sum).



\newpage
\vskip .4in
{\bf Spectral Radius}

The spectral radius $\rho(A)$ of a matrix $A$ is defined by

\begin{center}
	$\rho(A) = max |\lambda|$, where $\lambda$ is an eigenvalue of $A$
\end{center}


\vskip .4in
{\bf Jacobi Method}

The Jacobi method can be written in the form $x^{(k)} = T x^{(k-1)} + c$ by splitting $A$ into its diagonal and off-diagonal parts.
$$A = D - L - U$$
So we have,

\begin{center}
	$x^{(k)} = D^{-1}(L + U)x^{(k-1)}+D^{-1}b$, $k=1,2,....$
\end{center}

To shorten,

\begin{center}
	$T_j = D^{-1}(L + U)$ and $c_j = D^{-1}b$
\end{center}

So we have 

$$
x^{(k)} = T_jx^{(k-1)} + c_j
$$



\vskip .4in
{\bf Gauss-Siedel Method}

We have

$$x^{(k)} =(D-L)^{-1}Ux^{(k-1)}+(D-L)^{-1}b$$

Letting

\begin{center}
	$T_g = (D-L)^{-1}U$ and $c_g = (D-L)^{-1}b$
\end{center}

We get
$$x^{(k)} = T_gx_{(k-1)} + c_g$$
Note: for $D - L$ to be nonsingular, it is necessary and sufficient that $a_{ii} \neq 0$.


\vskip .4in
{\bf Conjugate Gradient Method}

Matrix must be positive definite.

input $A$, $b$, $x_0$ \\
$r_0 = b - Ax_0$ \\
DO $k = 1...n$ \\
\tab $\alpha_{k} = \frac { r^{t}_{k-1} r_{k-1}} {P^{t}_{k} A P^{t}}$

\end{document}