\documentclass[11pt]{article} % 

\setlength\parindent{0pt}

\begin{document}

\begin{center}
Definitions
\end{center}

\textbf{vector space}: A vector space $V$ over a field $F$ consists of a set on which two operations (called \textbf{addition} and \textbf{scalar multiplication}) are defined so that for each pair of elements $x,y \in V$ there is a unique element $x+y$ in $V$, and for each element $a$ in $F$ and each element $x$ in $V$ there is a unique elemnt $ax$ in $V$, such that the following conditions hold: ...

\vskip .5in

\textbf{subspace}: A subset $W$ of a vector space $V$ over a field $F$ is called a \textbf{subspace} of $V$ if $W$ is a vector space over $F$ with the operations of addition and scalar multiplication defined on $V$.

\vskip .5in

\textbf{linear combination}: Let $V$ be a vector space and $S$ a nonempty subset of $V$. A vector $v \in V$ is called a \textbf{linear combination} of vectors of $S$ if there exist a finite number of vectors $u_1,u_2,..,u_n$ in $S$ and scalars $a_1,a_2,...,a_n$ in $F$ such that $v = a_1 u_1 + a_2 u_2 + \cdots + a_n u_n$. In this case we also say that $v$ is a linear combination of $u_1,u_2,...,u_n$ and call $a_1,a_2,...,a_n$ the coefficients of the linear combination.

\vskip .5in

\textbf{span}: Let $S$ be a nonempty subset of a vector space $V$. The \textbf{span} of $S$, denoted $span(S)$, is the set consisting of all linear combinations of the vectors in $S$. For convenience, we define $span(0) = {0}$.

\vskip .5in

\textbf{linearly dependent}: A subset $S$ of a vector space $V$ is called \textbf{linearly depenedent} if there exist a finite number of distinct vectors $u_1,u_2,...,u_n$ in $S$ and scalars $a_1,a_2,...,a_n$, not all zero such that
$$a_1 u_1 + a_2 u_2 + \cdots + a_n u_n = 0$$

\vskip .5in

\textbf{basis}: A \textbf{basis} $\beta$ for a vector space $V$ is a linearly indepenedent subset of $V$ that generates $V$. If $\beta$ is a basis for $V$, we also say that the vectors of $\beta$ form a basiss for $V$.

\vskip .5in

\textbf{finite-dimensional}: A vector space is called \textbf{finite-dimensional} if it has basis consisting of a finite number of vectors. The unique number of vectors in each basis for $V$ is called the \textbf{dimension} of $V$ and is denoted by $dim(V)$. A vector space that is not finite-dimensional is called \textbf{inifinite-dimensional}.

\vskip .5in

\textbf{maximal}: Let $\mathcal{F}$ be a family of sets. A member $M$ of $\mathcal{F}$ (with respect to set inclusion) if $M$ is contained in no member of $\mathcal{F}$ other than $M$ itself.

\vskip .5in

\textbf{linear transformation}: Let $V$ and $W$ be vector spaces (over $F$). We call a function $T:V \rightarrow W$ a \textbf{linear transformation} from $V$ to $W$ if, for all $x,y \in V$ and $c \in F$, we have
\begin{itemize}
\item $T(x+y) = T(x) + T(y)$
\item $T(cx) = xT(x)$
\end{itemize}

\vskip .5in

\textbf{null space}: Let $V$ and $W$ be vector spaces, and let $T:V \rightarrow W$ be linear. We define the \textbf{null space} $N(T)$ of $T$ to be the set of all vectors in $V$ such that $T(x) = 0$; that is, $N(T)$; that is, $N(T) = {x \in V: T(x) = 0}$.

\vskip .5in

\textbf{range}: We define the \textbf{range} $R(T)$ of $T$ to be the subset of $W$ consisting of all images (under $T$) of vectors in $V$; that is , $R(T) = {T(x): x \in V}$.

\vskip .5in

\textbf{nullity}: $dim(N(T))$

\vskip .5in

\textbf{rank}: $dim(R(T))$

\vskip .5in



\end{document}