\documentclass[11pt]{article} % 

\setlength{\oddsidemargin}{-0.15in}
\setlength{\topmargin}{-0.5in}
\setlength{\textwidth}{6.5in}
\setlength{\textheight}{9in}
\setlength\parindent{0pt}

\newcommand{\cP}{{\cal P}}
\newcommand{\IN}{{\bf N}}
\newcommand{\IZ}{{\bf Z}}
\newcommand{\IR}{{\bf R}}
\newcommand{\IQ}{{\bf Q}}
\newcommand{\points}[1]{{\it (#1 Points)}}
\newcommand{\qed}{{\hfill {\rm QED}}}

\begin{document} 
%\maketitle

Kelvin Abrokwa-Johnson \\
14 February 2016 \\
Math 309

\begin{center} Homework 3 \end{center}



%----------------------------------------------------------------------------------------------------------------
%----------------------------------------------------------------------------------------------------------------
%----------------------------------------------------------------------------------------------------------------
%----------------------------------------------------------------------------------------------------------------


{\bf 1.3.12}

\medskip
{\bf Solution:}

Let $D$ be the set of diagonal matrices.

\begin{itemize}

\item The $0$ matrix is in $D$ because $A_{ij} = 0$ for all entries where $i > j$ (moreover, all entries are zero).

\item Take some scalar $c \in F$ and matrix $A \in D$. When we perform the multiplication $cA$, for every entry $A_{ij}$ where $i > j$ and thus $A_{ij} = 0$, $cA_{ij} = 0$. Thus $cA \in D$, i.e. $D$ is closed under scalar multiplication.

\item Take matrices $A, B \in D$. When we add $A$ and $B$, since $A_{ij} = B_{ij} = 0$ whenever $i > j$, $A_{ij} + B_{ij} = 0$ when those entries are below the diagonal. So $A + B \in D$, i.e. $D$ is closed under addition.
\end{itemize}


%----------------------------------------------------------------------------------------------------------------
%----------------------------------------------------------------------------------------------------------------
%----------------------------------------------------------------------------------------------------------------
%----------------------------------------------------------------------------------------------------------------

\vskip .5in
{\bf 1.6.16}

\medskip
{\bf Solution:}

A basis for $W$ is the set of $E^{ij}$ where the only non-zero entry is a $1$ in the $i,j$ position for every $E^{ij}$ where $i \leq j$.

$dim(W) = n + (n - 1) + \cdots + 1 = \frac{1}{2}n(n+1)$


%----------------------------------------------------------------------------------------------------------------
%----------------------------------------------------------------------------------------------------------------
%----------------------------------------------------------------------------------------------------------------
%----------------------------------------------------------------------------------------------------------------

\vskip .5in
{\bf 1.6.22}

\medskip
{\bf Solution:}

A neccessary and sufficient condition is that $W_1 \subseteq W_2$.
\begin{itemize}
\item $W_1 \subseteq W_2 \rightarrow dim(W_1 \cap W_2) = dim(W_1)$

If $W_1 \subseteq W_2$ then $W_1 \cap W_2 = W_1$. So, since we assumed $W_1 \subseteq W_2$, it follows that $dim(W_1 \cap W_2) = dim(W_1)$.

\item $dim(W_1 \cap W_2) = dim(W_1) \rightarrow W_1 \subseteq W_2$

Let $dim(W_1 \cap W_2) = n$, $dim(W_1) = n$, and $dim(W_2) = m$. Now we define some arbitrary basis for $W_1 \cap W_2$, $\mathcal{B} = \{ v_1, v_2, ..., v_n \}$. Since $\mathcal{B}$ is a basis it is linearly independent. We also know that $W_1 \cap W_2 \in W_1$, so $\mathcal{B} \in W_1$. Since $\mathcal{B}$ has $n$ linearly independent elements and $W_1$ had a dimension of $n$ it forms a maximal linearly independent subset of $W_1$, so $\mathcal{B}$ is a basis for $W_1$. Now, we also know that $W_1 \cap W_2 \subseteq W_2$, so its basis, $\mathcal{B}$ can be extended into a basis $W_2$. Since the basis for $W_2$ contains the basis for $W_1$ we can conclude that $W_1 \subseteq W_2$.

\end{itemize}


%----------------------------------------------------------------------------------------------------------------
%----------------------------------------------------------------------------------------------------------------
%----------------------------------------------------------------------------------------------------------------
%----------------------------------------------------------------------------------------------------------------

\vskip .5in
{\bf 1.6.29.a}

\medskip
{\bf Solution:}

Let $dim(W_1 \cap W_2) = k$, $dim(W_1) = n$, and $dim(W_2) = m$. Then we may express a basis for $W_1 \cap W_2$ as some linearly independent subset $\mathcal{B}_3 = \{ u_1, u_2, ..., u_k \}$. Since, $W_1 \cap W_2 \subseteq W_1, W_2$ we may extend $\mathcal{B_3}$ to bases for $W_1$ and $W_2$. Let $\mathcal{B}_1 = \mathcal{B}_3 \cup \{ v_1, ..., v_{m-k} \}$ be a basis for $W_1$ and $\mathcal{B}_2 = \mathcal{B}_3 \cup \{ w_1,...w_{n-k} \}$ be a basis for $W_2$. Now we can show that the set $\mathcal{B}_3 \cup \mathcal{B}_1 \setminus \mathcal{B}_3 \cup \mathcal{B}_2 \setminus \mathcal{B}_3$ is a basis for $W_1 + W_2$. First lets show that the set is linearly independent. First we know that each part of the union is linearly independent because they were each subsets of bases. Now, take $v \in span(\mathcal{B}_1 \setminus \mathcal{B}_3)$. We know that $v \not\in span(\mathcal{B}_2 \setminus \mathcal{B}_3)$ because those portions of the bases didn't share any elements in common. Likewise, for any $v \in span(\mathcal{B}_2 \setminus \mathcal{B}_3)$, $v \not\in span(\mathcal{B}_1 \setminus \mathcal{B}_3)$. Now take $v \in span(\mathcal{B}_3)$, we immediately see that $v \notin \mathcal{B}_1 \setminus \mathcal{B}_3$ and $v \notin \mathcal{B}_2 \setminus \mathcal{B}_3$. So the set $\mathcal{B}_3 \cup \mathcal{B}_1 \setminus \mathcal{B}_3 \cup \mathcal{B}_2 \setminus \mathcal{B}_3$ is linearly independent. Now we can show that the set $\mathcal{B} = \mathcal{B}_3 \cup \mathcal{B}_1 \setminus \mathcal{B}_3 \cup \mathcal{B}_2 \setminus \mathcal{B}_3$ spans $W_1 + W_2$. We can show this by contradiction. Take $v \in W_1 + W_2$. If $v \not\in span(\mathcal{B})$ then $v \not\in span(W_1)$ and $v \not\in span(W_2)$, so $v \not\in W_1 + W_2$. Which is a contradiction. So $\mathcal{B}$ spans $W_1 + W_2$.




%----------------------------------------------------------------------------------------------------------------
%----------------------------------------------------------------------------------------------------------------
%----------------------------------------------------------------------------------------------------------------
%----------------------------------------------------------------------------------------------------------------

\vskip .5in
{\bf 1.6.31}

\medskip
{\bf Solution:}

{\bf a.} Let $dim(W_1 \cap W_2) = k$. Then we can define some arbitrary basis $\mathcal{B} = \{ u_1, u_2,..., u_k \}$ for $W_1 \cap W_2$. Since $W_1 \cap W_2$ is a subspace of $W_2$, $\mathcal{B} \subseteq W_2$. Since $\mathcal{B}$ is linearly independent, a maximal linearly independent subset of $W_2$ must have at least k elements. So $k  = dim(W_1 \cap W_2) \leq n$.

{\bf b.} We have shown that $dim(W_1 + W_2) = dim(W_1) + dim(W_2) - dim(W_1 \cap W_2)$. Since $dim(W_1 \cap W_2) \geq 0$, $dim(W_1 + W_2) \leq dim(W_1) + dim(W_2) = m + n$.




%----------------------------------------------------------------------------------------------------------------
%----------------------------------------------------------------------------------------------------------------
%----------------------------------------------------------------------------------------------------------------
%----------------------------------------------------------------------------------------------------------------

\vskip .5in
{\bf 1.6.32}

\medskip
{\bf Solution:}

{\bf a.} We let $W_1 = R^3$ and $W_2 = span(\{(1,0,0)\})$. So $m = 3$ and $n = 1$. $W_1 \cap W_2 = W_2$ since $W_2 \subset W_1$. So, $dim(W_1 \cap W_2) = dim(W_2) = 1 = n$.

{\bf b.} Let $W_1 = span(\{(1,0,0), (0,1,0)\})$ and $W_2 = span(\{(0,0,1)\})$ ($m = 2, n = 1$). So $dim(W_1 + W_2) = dim(span(\{(1, 0, 0), (0, 1, 0), (0, 0, 1)\})) = dim(R^3) = m + n = 3$.

{\bf c.} Let $W_1 = span(\{(1,0,0), (0,1,0)\})$ and $W_2 = span(\{ (0,1,0), (0,0,1) \})$. So $m = dim(W_1) = 2$, $n = dim(W_2)$, and $dim(W_1 \cap W_2) = 1$ . So $dim(W_1 + W_2) = m + n - 1 < m + n$.



%----------------------------------------------------------------------------------------------------------------
%----------------------------------------------------------------------------------------------------------------
%----------------------------------------------------------------------------------------------------------------
%----------------------------------------------------------------------------------------------------------------

\vskip .5in
{\bf 1.7.4}

\medskip
{\bf Solution:}

Let $\mathcal{B} = \{ u_1, u_2, ..., u_n \}$ be any basis for $W$. By Theorem 1.13, there exists a maximal linearly independent subset of $V$, lets call it $\mathcal{S}$, that contains $\mathcal{B}$. Since $\mathcal{S}$ is a basis for $V$ (see Theorem 1.12), and $\mathcal{B} \subseteq \mathcal{S}$, then $\mathcal{B}$ is a subset of a basis for $V$.


\end{document}