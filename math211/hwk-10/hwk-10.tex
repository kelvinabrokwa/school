\documentclass[11pt]{article} % 

\usepackage{amsfonts,amstext}
\usepackage{amsthm}

\setlength{\oddsidemargin}{-0.15in}
\setlength{\topmargin}{-0.5in}
\setlength{\textwidth}{6.5in}
\setlength{\textheight}{9in}

\newcommand{\cP}{{\cal P}}
\newcommand{\IN}{{\bf N}}
\newcommand{\IZ}{{\bf Z}}
\newcommand{\IR}{{\bf R}}
\newcommand{\IQ}{{\bf Q}}
\newcommand{\points}[1]{{\it (#1 Points)}}


\begin{document} 
%\maketitle

\noindent
{\Large Math 211 \hfill Homework 10 \hfill Kelvin Abrokwa-Johnson}

\medskip
\noindent{\bf 1.} (10 points)
Show that 
$${\mathbb B} = \left\{1, 1-t, 2-4t+t^2, 6-18t+9t^2-t^3\right\}$$
is a basis for $\mathbb{P}_3$, and find the change of basis matrix from 
${\mathbb B} $ to ${\mathbb C} = \{1, t, t^2, t^3\}$ and the change of the basis 
matrix from ${\mathbb C} $ to ${\mathbb B}$
Find $[u]_{\mathbb B}$ for $u = 1 - 3t + 9t^2 -t^3$.

We can represent the polynomial as a set of vectors:

$$
\left\{
	\left[ \matrix{ 1 \cr 0 \cr 0 \cr 0 }\right],
	\left[ \matrix{ 1 \cr -1 \cr 0 \cr 0 }\right],
	\left[ \matrix{ 2 \cr -4 \cr 1 \cr 0 }\right],
	\left[ \matrix{ 6 \cr -18 \cr 9 \cr 1 }\right]
\right\}
$$

In this form it is clear that the set is linearly independent, therefore the set is linearly independent and is a basis for $\mathbb{P}_3$.

The change of basis matrix from $\mathbb{C}$ to $\mathbb{B}$ is composed of the vectors above:
$$
\left[
	\matrix{
		1 & 1 & 2 & 6 \cr
		0 & -1 & -4 & -18 \cr
		0 & 0 & 1 & 9 \cr
		0 & 0 & 0 & 1 \cr
	}
\right]
$$

$$
[u]_{\mathbb{B}} =
\left[
	\matrix{
		1 & 1 & 2 & 6 \cr
		0 & -1 & -4 & -18 \cr
		0 & 0 & 1 & 9 \cr
		0 & 0 & 0 & 1 \cr
	}
\right]
\left[ \matrix{ 1 \cr -3 \cr 9 \cr -1 } \right] = 
\left[ \matrix{ 22 \cr -51 \cr 18 \cr -1 } \right]
$$





\medskip
\noindent{\bf 2.} (5 points)
 Find the rank, the nullity (dimension of null space), a basis for the 
column space, a basis for the row space of the following matrix
$$\left[\matrix{1&1&-2&0&1&0\cr 0&1&-1&0&-3&0\cr 0&0&1&1&-13&0\cr 
0&0&0&0&1&0\cr 0&0&0&0&0&1}\right].$$

rank $A$ = $5$

dim Nul $A$ = $n - rank A$ = $1$

Basis Col $A$ = $\left\{
	\left[ \matrix{ 1 \cr 0 \cr 0 \cr 0 \cr 0 } \right],
	\left[ \matrix{ 1 \cr 1 \cr 0 \cr 0 \cr 0 } \right],
	\left[ \matrix{ -2 \cr -1 \cr 1 \cr 0 \cr 0 } \right],
	\left[ \matrix{ 1 \cr -3 \cr -13 \cr 1 \cr 0 } \right],
	\left[ \matrix{ 0 \cr 0 \cr 0 \cr 0 \cr 1 } \right]
\right\}$ 

Basis for row space $ = \left\{
	\left[ \matrix{ 1 \cr 1 \cr -2 \cr 0 \cr 1 \cr 0 } \right],
	\left[ \matrix{ 0 \cr 1 \cr -1 \cr 0 \cr -3 \cr 0 } \right],
	\left[ \matrix{ 0 \cr 0 \cr 1 \cr 1 \cr -13 \cr 0 } \right],
	\left[ \matrix{ 0 \cr 0 \cr 0 \cr 0 \cr 1 \cr 0 } \right],
	\left[ \matrix{ 0 \cr 0 \cr 0 \cr 0 \cr 0 \cr 1 } \right]
\right\}$



\medskip\noindent
{\bf 3.}
Find the characteristic polynomial for the matrix and two linearly independent 
eigenvectors for the matrix 
$$A=\left[\matrix{4 & 4\cr 1&4}\right].
$$



\medskip\noindent
{\bf 4.} Show that $A = \left[\matrix{1 & 2 & 0 \cr 0 & 1 & 3 \cr 0 & 0 & 1}\right]$
has at most one linear independent eigenvector.

$A - \lambda I = \left[
	\matrix{
		1 - \lambda & 2 & 0 \cr
		0 & 1 - \lambda & 3 \cr
		0 & 0 & 1 - \lambda
	}
\right]$
so $det A = (1- \lambda)^3$, and the matrix is singular when $\lambda = 1$. We substitute this eigenvalue back into $A - \lambda I$ to get $
\left[
	\matrix{
		0 & 2 & 0 \cr
		0 & 0 & 3 \cr
		0 & 0 & 0	
	}
\right]
$. We set this matrix to $0$ and get the solution $Span \left\{ \left[ \matrix{ 1 \cr 0 \cr 0 } \right] \right\}$ and this is the only eigenvector.


\medskip
\noindent{\bf 5.}
Find the characteristic polynomial for the matrix and three linearly independent 
eigenvectors for the matrix 
$$A=\left[\matrix{4&0&-1\cr 0&3&4\cr 0&0&2}\right].
$$
 
\medskip
\noindent{\bf 6.}
Find the values $h$ so that  there are two linearly independent
eigenvectors corresponding to the eigenvalue $\lambda =4$
 for the matrix
$$A=\left[\matrix{4&2&3&3\cr 0&2&h&3\cr 0&0&4&14\cr 0&0&0&2}\right].$$

[Hint: Find $h$ such that the null space of $A-4I_4$ has nullity 2.]


\medskip\noindent
{\bf 7.} Show that $A$ and $A^T$ have the same characteristic polynomial.

[Hint: Show that $\det(A- \lambda I) = \det(A^T - \lambda I)$.]

\end{document}
