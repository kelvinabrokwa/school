\documentclass[11pt]{article} % 

\usepackage{amsfonts,amstext}
\usepackage{amsthm}

\setlength{\oddsidemargin}{-0.15in}
\setlength{\topmargin}{-0.5in}
\setlength{\textwidth}{6.5in}
\setlength{\textheight}{9in}

\newcommand{\cP}{{\cal P}}
\newcommand{\IN}{{\bf N}}
\newcommand{\IZ}{{\bf Z}}
\newcommand{\IR}{{\bf R}}
\newcommand{\IQ}{{\bf Q}}
\newcommand{\points}[1]{{\it (#1 Points)}}


\begin{document} 
%\maketitle

\noindent
{\Large Math 211 \hfill Homework 10 \hfill Kelvin Abrokwa-Johnson}

\medskip
\noindent{\bf 1.} (10 points)
$$0 = \gamma_1(1) + \gamma_2(1 - t) + \gamma_3(2 - 4t + t^2) + \gamma_4(6 - 18t + 9t^2 +t^3)$$
$$0 = (-\gamma_4)t^3 + (9\gamma_4 + \gamma_3)t^2 + (-18\gamma_4 - 4\gamma_3 - \gamma_2)t + (\gamma_4 - 2\gamma_3 + \gamma_2 + \gamma_1)$$

Since we know $b \neq 0$ then we must show that $\gamma_1 = \gamma_2 = \gamma_3 = \gamma_4 = 0$. When we solve the system composed of the above coefficients set to $0$ we can clearly see below that the matrix has 4 pivoting columns so the system must only have one solution, the trivial solution.
$$
\left[
	\matrix{
		1 & 1 & 2 & 6 \cr
		0 & -1 & -4 & -18 \cr
		0 & 0 & 1 & 9 \cr
		0 & 0 & 0 & -1 \cr
	}
\right]
$$

Since we have shown that $\gamma_1 = \gamma_2 = \gamma_3 = \gamma_4 = 0$ we conclude that the set is also a basis.

The change of basis matrix from $\mathbb{C}$ to $\mathbb{B}$ is:
$$
\left[
	\matrix{
		1 & 1 & 2 & 6 \cr
		0 & -1 & -4 & -18 \cr
		0 & 0 & 1 & 9 \cr
		0 & 0 & 0 & -1 \cr
	}
\right]
$$

$$
[u]_{\mathbb{B}} =
\left[
	\matrix{
		1 & 1 & 2 & 6 \cr
		0 & -1 & -4 & -18 \cr
		0 & 0 & 1 & 9 \cr
		0 & 0 & 0 & 1 \cr
	}
\right]
\left[ \matrix{ 1 \cr -3 \cr 9 \cr -1 } \right] = 
\left[ \matrix{ 10 \cr -15 \cr 0 \cr 1 } \right]
$$




\medskip
\noindent{\bf 2.} (5 points)

rank $A$ = $5$

dim Nul $A$ = $n - rank A$ = $1$

Basis Col $A$ = $\left\{
	\left[ \matrix{ 1 \cr 0 \cr 0 \cr 0 \cr 0 } \right],
	\left[ \matrix{ 1 \cr 1 \cr 0 \cr 0 \cr 0 } \right],
	\left[ \matrix{ -2 \cr -1 \cr 1 \cr 0 \cr 0 } \right],
	\left[ \matrix{ 1 \cr -3 \cr -13 \cr 1 \cr 0 } \right],
	\left[ \matrix{ 0 \cr 0 \cr 0 \cr 0 \cr 1 } \right]
\right\}$ 

Basis for row space $ = \left\{
	(1, 1, -2, 0, 1, 0),
	(0, 1, -1, 0, -3, 0),
	(0, 0, 1, 1, -13, 0)
	(0, 0, 0, 0, 1, 0)
	(0, 0, 0, 0, 0, 1)
\right\}$



\medskip\noindent
{\bf 3.}
To find the characteristic polynomial we simply set the determinant of $\left[ \matrix{
	4 - \lambda & 4 \cr
	1 & 4 - \lambda	
} \right]$ equal to 0 and solve. We find that it is $(4 - \lambda)^2 - 4 = (\lambda - 2)(\lambda - 6) = 0$. We plug each of the solutions into the matrix $A - \lambda I$ and solve for $0$ to get $Span \left\{ \left[ \matrix{ -2 \cr 1 } \right] \right\}$ when $\lambda = 2$ and $Span \left\{ \left[ \matrix{ 2 \cr 1 } \right] \right\}$ when $\lambda = 6$. So two linearly independent eigenvectors are $\left[ \matrix{ -2 \cr 1 } \right]$ and $\left[ \matrix{ 2 \cr 1 } \right]$.


\medskip\noindent
{\bf 4.}
$A - \lambda I = \left[
	\matrix{
		1 - \lambda & 2 & 0 \cr
		0 & 1 - \lambda & 3 \cr
		0 & 0 & 1 - \lambda
	}
\right]$
so $det A = (1- \lambda)^3$, and the matrix is singular when $\lambda = 1$. We substitute this eigenvalue back into $A - \lambda I$ to get $
\left[
	\matrix{
		0 & 2 & 0 \cr
		0 & 0 & 3 \cr
		0 & 0 & 0	
	}
\right]
$. We set this matrix to $0$ and get the solution $Span \left\{ \left[ \matrix{ 1 \cr 0 \cr 0 } \right] \right\}$ and this is the only eigenvector.


\medskip
\noindent{\bf 5.}
Since this is a triangular matrix we can immediately see that the determinant is the product is the product of the diagonal entries. So the characteristic polynomial is $(4 - \lambda)(3 - \lambda)(2 - \lambda) = 0$. When we substitute each of these into the matrix $A - \lambda I$ we get the solutions $Span \left\{ \left[ \matrix{ 1 \cr 0 \cr 0 } \right] \right\}$, $Span \left\{ \left[ \matrix{ 0 \cr 1 \cr 0 } \right] \right\}$, and $Span \left\{ \left[ \matrix{ 1 \cr -8 \cr 2 } \right] \right\}$ so three linearly indepenedent eigenvectors are $\left[ \matrix{ 1 \cr 0 \cr 0 } \right]$, $\left[ \matrix{ 0 \cr 1 \cr 0 } \right]$, and $\left[ \matrix{ 1 \cr -8 \cr 2 } \right]$.
 
 
 
 
\medskip
\noindent{\bf 6.}
$A - 4 I_4 = \left[
	\matrix{
		0&2&3&3\cr
		0&-2&h&3\cr
		0&0&0&14\cr
		0&0&0&-2
	}
\right]$. After row reduction, we have $\left[
	\matrix{
		0 & 0 & h+3 & 0 \cr
		0 & 1 & h & 0 \cr
		0 & 0 & 0 & 0 \cr
		0 & 0 & 0 & 1
	}
\right]$ and we see that the null space of $A - 4 I_4$ has nullity 2 when $h = -3$.





\medskip\noindent
{\bf 7.} 
We know that $(A + B)^T = A^T + B^T$ and det $A$ = det $A^T$. So $det(A - \lambda I) = det(A - \lambda I)^T = det(A^T - (\lambda I)^T) = det(A^T - \lambda I)$.

\end{document}
