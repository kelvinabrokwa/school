\documentclass[10pt]{article} % 

\usepackage{amsfonts,amstext}
\usepackage{amsthm}

\setlength{\oddsidemargin}{-0.15in}
\setlength{\topmargin}{-0.5in}
\setlength{\textwidth}{6.5in}
\setlength{\textheight}{9in}

\newcommand{\cP}{{\cal P}}
\newcommand{\IN}{{\bf N}}
\newcommand{\IZ}{{\bf Z}}
\newcommand{\IR}{{\bf R}}
\newcommand{\IQ}{{\bf Q}}
\newcommand{\points}[1]{{\it (#1 Points)}}


\begin{document} 
\openup .7 \jot
%\maketitle

\noindent
{\Large Math 211 \qquad \qquad \qquad Homework 7 \hfill Kelvin Abrokwa-Johnson}

\medskip
\noindent{\bf 1.}
$$
det(A) = \left|
	\matrix{
		-5 & 3 \cr
		3 & -1	
	}
\right| = -4
$$
$$
x_1 = \frac{\left|
	\matrix{
		 9 & 3 \cr
		 -5 & -1	
	}
\right|}{det(A)} = \frac{6}{-4} = -\frac{3}{2}
$$
$$
x_1 = \frac{\left|
	\matrix{
		 -5 & 9 \cr
		 3 & -5
	}
\right|}{det(A)} = \frac{-2}{-4} = \frac{1}{2}
$$


\medskip
\noindent{\bf 2.}
$$
det(A) = \left|
	\matrix{
		2 & 1 & 1 \cr
		-1 & 0 & 3 \cr
		3 & 1 & 3	
	}
\right| = 5
$$

$$
det(A_1(b)) = \left|
	\matrix{
		4 & 1 & 1 \cr
		2 & 0 & 3 \cr
		2 & 1 & 3	
	}
\right| = -10, x_1 = \frac{-10}{5} = -2
$$
$$
det(A_1(b)) = \left|
	\matrix{
		2 & 4 & 1 \cr
		-1 & 2 & 3 \cr
		2 & 2 & 3	
	}
\right| = 40, x_2 = \frac{40}{5} = 8
$$
$$
det(A_1(b)) = \left|
	\matrix{
		2 & 1 & 4 \cr
		-1 & 0 & 2 \cr
		3 & 1 & 2
	}
\right| = 0, x_3 = \frac{0}{5} = 0
$$

\medskip
\noindent{\bf 3.}
$$
A_{11} = 0, A_{12} = -3, A_{13} = 3, A_{21} = 1, A_{22} -1, A_{23} = 2, A_{31} = 0, A_{32} = -3, A_{33} = 6
$$
$$
det(A) = \left|
	\matrix{
		0&-2&-1\cr
		3&0&0\cr
		-1&1&1
	}
\right| = 3
$$
$$
Adj(A) = \left[
	\matrix{
		0 & 1 & 0 \cr
		-3 & -1 & -3 \cr
		3 & 2 & 6	
	}
\right]
$$
$$
A^{-1} = \left[
	\matrix{
		0 & \frac{1}{3} & 0 \cr
		-1 & -\frac{1}{3}	& -1 \cr
		1 & \frac{2}{3} & 2
	}
\right]
$$


\medskip\noindent
{\bf 4.}
We translate the parallelogram to the origin by adding $(0, 2)$ to all of the vertices to get:
$$
(0, 0), (6, 1), (-3, 3), (3, 4)
$$
So our matrix $A = \left[\matrix{ 6 & 1 \cr -3 & 3 }\right]$. The area of the parrallelogram $A = det(A) = 21.$


\pagebreak
\medskip
\noindent{\bf 5.}
\begin{enumerate}
\item If $xy \geq 0$ then $x$ and $y$ are both positive or both negative (or one or both is zero), that is, $x$ and $y$ have the same sign for any $u \in W$. So, the signs of $x$ and $y$ will always remain the same when multiplied by some  $c \in \mathbb{R}$ therefore $xy$ will be greater than or equal to 0.

\item $$u = \left[\matrix{ -1 \cr -2 }\right], v = \left[\matrix{ 2 \cr 1 }\right], u + v = \left[\matrix{ 1 \cr -1 }\right] \notin W, 1 \cdot -1 \leq 0$$
\end{enumerate}




\medskip
\noindent{\bf 6.}
$W$ can be written as $s\left[\matrix{  2 \cr 2 \cr 2 \cr 0 }\right] + t\left[\matrix{  4 \cr 0 \cr -3 \cr 5 }\right]$ so $W = Span\left\{ \left[\matrix{  2 \cr 2 \cr 2 \cr 0 }\right], \left[\matrix{  4 \cr 0 \cr -3 \cr 5 }\right] \right\}$. Therefore, $W$ is a subspace by Theorem 1.



\medskip
\noindent{\bf 7.}
We know that $0 \notin W$ because $0 + 0 + 0 \neq 1$ so $W$ is not a vector space.




\medskip
\noindent{\bf 8.}
(a) It is clear what $F0_{2x4} = 0$ so $0_{2x4} \in W$.

\noindent(b) $X, Y \in W \rightarrow FX = 0, FY = 0$. So, $F(X+Y) = FX + FY = 0 + 0 = 0 \in W$. Therefore $ X + Y \in W$.

\noindent(c) If $X \in W$ then $FX = 0$, so to show $cX \in W$ we simply evaluate $FcX = c(FX)  = c(0) = 0 \in W$. Therefore $cX \in W$.


\medskip\noindent
{\bf 9.} 
(a) $H$ and $K$ are subspaces implies that $\exists x \in H, \exists y \in K : x = 0, y = 0$ therefore $x + y = 0 + 0 = 0$. So $0 \in H + K$.

\noindent(b) Say $u = h_1 + k_1, v = h_2 + k_2$ for $h_1, h_2 \in H, k_1, k_2 \in K$, then $u + v = (h_1 + h_2) + (k_1 + k_2)$. Since $H, K$ are subspaces, we know that $h_1 + h_2 = h \in H$ and $k_1 + k_2 = k \in K$, so $u + v = h + k \in H + K$.

\noindent(c) Say $u = h_1 + k_1 \in H + K$, then $cu = c(h_1 + k_1) = ch_1 + ck_1$. Since $h_1 \in H \rightarrow ch_1 = h \in H$ and $k_1 \in K \rightarrow ck_1 = k \in K$, it is clear that $cu = h + k \in H + K$.

\end{document}

