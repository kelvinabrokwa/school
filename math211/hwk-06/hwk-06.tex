\documentclass[11pt]{article} % 

\usepackage{amsfonts,amstext}
\usepackage{amsthm}

\setlength{\oddsidemargin}{-0.15in}
\setlength{\topmargin}{-0.5in}
\setlength{\textwidth}{6.5in}
\setlength{\textheight}{9in}

\newcommand{\cP}{{\cal P}}
\newcommand{\IN}{{\bf N}}
\newcommand{\IZ}{{\bf Z}}
\newcommand{\IR}{{\bf R}}
\newcommand{\IQ}{{\bf Q}}
\newcommand{\points}[1]{{\it (#1 Points)}}


\begin{document}
\openup .3\jot 
%\maketitle

\medskip
\centerline{Five points for each question.}

\medskip
\noindent
{\Large Math 211 \qquad Homework 6
\hfill Kelvin Abrokwa-Johnson}

\medskip
\noindent{\bf 1.}
$$
det(A) =
8 \left|\matrix{ 0 & 3 \cr -2 & 5 }\right| -
1 \left|\matrix{ 4 & 3 \cr 3 & 5 }\right| +
6 \left|\matrix{ 4 & 0 \cr 3 & -2 }\right| =
8(6) - 11 - 6(8) = -11
$$

\medskip
\noindent{\bf 2. }
$$
det(A) = 4 \left|\matrix{
	-1 & 0 & 0 \cr
    6 & 3 & 0 \cr
    -8 & 4 & -3
}\right| =
4(-1)\left|\matrix{
	3 & 0 \cr
    4 & -3
}\right| =
4 \times -1 \times 3 \times \left|-3\right| = 36
$$


\medskip
\noindent{\bf 3.}
We can first augment the matrix as a visual aid like so:
$$
\left[
	\matrix{
    	1 & 3 &5 & 1 & 3 \cr
        2 & 1 & 1 & 2 & 1 \cr
        3 & 4 & 2 & 3 & 4
    }
\right]
$$
Ans now we add and products of the downward diagonals and subtract the products of the upward diagonals. So,
$$
det(A) = 40 + 2 + 9 -15 - 4 -12 = 51 - 31 = 20
$$


\medskip
\noindent{\bf 4.}
$$
det(A) = 3\left|\matrix{
	1 & 5 & -3 \cr
    0 & -6 & 4 \cr
    0 & 3 & -1
}\right| =
3(2) \left|\matrix{
	1 & 5 & -3 \cr
    0 & -3 & 2 \cr
    0 & 0 & 1
}\right| = -18
$$

\medskip
\noindent{\bf 5.}
We begin our row reduction with the first column. So,
$$
det(A) = \left|
	\matrix{
    	1&3&3&-4\cr
        0&1&2&-5\cr
        2&5&4&-3\cr
        -3&-7&-5&2
	}
\right| =
\left|
	\matrix{
    	1 & 3 & 3 & -4 \cr
        0 & 1 & 2 & -5 \cr
        0 & -1 & -2 & 5 \cr
        0 & 2 & 4 & -10
	}
\right|
$$
We immediately see that the last three rows are multiples of each other, therefore, further reduction would result in $0$ rows. We can conclude that the $det(A) = 0$.


\medskip
\noindent{\bf 6.}
$det(M) = 7$ because it takes at least 2 row swaps to reach $M$ from the original matrix, call it $A$. so $(M) = -1(-1)det(A) = 7$.

Since $N$ is obtained by adding a multiple of one row to another the determinant is the same as the original matrix. $det(N) = 7$.


\medskip\noindent{\bf 7.}
(a) We know that $A = IA$ so $det(A) = det(IA)$. Therefore it follows that $det(rA) = det(rI)det(A)$. The determinant of $rI$ can be determined by the product of the diagonal which in this case would simply be $r^n$. So $det(rA) = r^ndet(A)$.

(b) $det(A) \neq 0$ means that $A$ is invertible, therefore, there exists a set of matrices $E_1 \cdot\cdot\cdot E_k$ such that $E_k \cdot\cdot\cdot E_1 A = I_n$. We know that $1 = det(I_n) = det(E_k) \cdot \cdot \cdot det(E_1) det(A)$. So $\frac{1}{det(A)} = det(E_k) \cdot \cdot \cdot det(E_1) = det(A^{-1})$. QED.

\medskip
\noindent{\bf 8.}
We start by expanding the expression to $det(PAP^{-1}) = det(P)det(A)det(P^{-1})$. Here there are two cases. If $A$ is not invertible, then $det(A) = 0$ and so $det(P) \cdot 0 \cdot det(P^{-1}) = 0 = det(A)$. For second case, in which $A$ is invertible, we can simple the expression as $det(P)det(A)\frac{1}{det(P)} = 1\cdot det(A) = det(A)$.

\medskip
\noindent{\bf 9.}

(a) $det(AB) = det(A)det(B) = -2$

(b) $det(B^5) = det(B)^5 = -32$

(c) $det(2A) = 2^4det(A) = -16$

(d) $det(A^TA) = det(A)^2 = 1$

(e) $det(AB^{-1}) = \frac{det(A)}{det(B)} = -\frac{1}{2}$




\end{document}
