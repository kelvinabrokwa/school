\documentclass[11pt]{article} % 

\usepackage{amsfonts,amstext}

\setlength{\oddsidemargin}{-0.15in}
\setlength{\topmargin}{-0.5in}
\setlength{\textwidth}{6.5in}
\setlength{\textheight}{9in}

\newcommand{\cP}{{\cal P}}
\newcommand{\IN}{{\bf N}}
\newcommand{\IZ}{{\bf Z}}
\newcommand{\IR}{{\bf R}}
\newcommand{\IQ}{{\bf Q}}
\newcommand{\points}[1]{{\it (#1 Points)}}
\newcommand{\qed}{{\hfill {\rm QED}}}

\begin{document} \openup .5 \jot
%\maketitle

\noindent
{\Large Math 211 \quad  Homework 4 \hfill Name:  Kelvin Abrokwa-Johnson}

\medskip
\centerline{Five points for each question unless specified otherwise.}

\medskip
\noindent{\bf 1.}
$$
AB = 
\left[\matrix{4 & -3\cr -3 & 5\cr 0 & 1}\right]
\left[\matrix{1&4\cr 3&-2}\right] = 
\left[
	\matrix{
		-5 & 22 \cr
		12 & -22 \cr
		3 & -2	
	}
\right]
$$





\medskip
\noindent{\bf 2.}
$$
A =
\left[
	\matrix{
		1 & 0 & 0 \cr
		0 & 0 & 0 \cr
		0 & 0 & 0
	}
\right],
B =
\left[
	\matrix{
		0 & 0 & 0 \cr
		0 & 1 & 0 \cr
		0 & 0 & 0
	}
\right],
C =
\left[
	\matrix{
		0 & 0 & 0 \cr
		0 & 0 & 0 \cr
		0 & 0 & 1
	}
\right]
$$







\medskip
\noindent{\bf 3.}
(a)
$$
A =
\left[
	\matrix{
		7 & 3 \cr
		-6 & -3	
	}
\right],
B = 
\left[
	\matrix{
	-9 \cr 4	
	}
\right]
$$

(b)
$$
\left[
	\matrix{
		7 & 3 & 1 & 0 \cr
		-6 & -3 & 0 & 1	
	}
\right] \rightarrow
\left[
	\matrix{
		1 & 0 & 1 & 1 \cr
		-6 & -3 & 0 & 1	
	}
\right] \rightarrow
\left[
	\matrix{
		1 & 0 & 1 & 1 \cr
		0 & -3 & 6 & 7	
	}
\right] \rightarrow
\left[
	\matrix{
		1 & 0 & 1 & 1 \cr
		0 & 1 & -2 & -\frac{7}{3}
	}
\right]
$$

$$
A^{-1} =
\left[
	\matrix{
		1 & 1 \cr
		-2 & -\frac{7}{3}	
	}
\right]
$$

(c)
$$
A^{-1}b =
\left[
	\matrix{
		1 & 1 \cr
		-2 & -\frac{7}{3}	
	}
\right]
\left[
	\matrix{
	-9 \cr 4	
	}
\right] = 
\left[
	\matrix{
		-5 \cr \frac{26}{3}	
	}
\right]
$$

$$
Ax_0 =
\left[
	\matrix{
		7 & 3 \cr
		-6 & -3	
	}
\right]
\left[
	\matrix{
		-5 \cr \frac{26}{3}	
	}
\right] =
\left[
	\matrix{
		-9 \cr 4	
	}
\right] =
b
$$







\medskip 
\noindent{\bf 4.} Let $A$ be a $3\times 3$ matrix.
Determine all possible reduced row echelon form of the augmented 

matrix $[A \ I_3]$ 
and decide whether $A$ is invertible in each of the following cases.

(a) The matrix $A$ has 1 pivoting column.

(b) The matrix $A$ has 2 pivoting columns.

(c) The matrix $A$ has 3 pivoting columns.

For the Augmented matrix $\left[\matrix{ A & I_3 }\right]$ we can say use the the variables $S$ and $T$ for each section such that $\left[\matrix{ A & I_3 }\right] = \left[\matrix{ S & T }\right]$.

(a)
$$
S = \left\{
\left[
	\matrix{
		1 & * & * \cr
		0 & 0 & 0 \cr
		0 & 0 & 0	
	}
\right],
\left[
	\matrix{
		1 & * & * \cr
		0 & 0 & 0 \cr
		0 & 0 & 0	
	}
\right],
\left[
	\matrix{
		1 & * & * \cr
		0 & 0 & 0 \cr
		0 & 0 & 0	
	}
\right]
\right\}
$$





\medskip\noindent
{\bf 5.} Using regular matrix algebra we see:

$$AB = AC$$
$$AB - AC = 0$$
$$A(B - C) = 0$$
$$A^{-1}A(B - C) = A^{-1}0$$
$$B - C = 0$$
$$B = C$$




\medskip
\noindent{\bf 6.}
Let 
$A = \left[\matrix{1 & -2 & 1 \cr 4 & -7 & 3 \cr -2 & 6 & -3\cr}\right]$.
Apply row reduction to the matrix  $\left[\matrix{A & I_3}\right]$ 
until we have the 

reduced row echelon form $\left[\matrix{I_3& B }\right]$,
and verify that $AB = I_3$.

$$
\left[
	\matrix{
		1 & -2 & 1 & 1 & 0 & 0 \cr
		4 & -7 & 3 & 0 & 1 & 0 \cr
		-2 & 6 & -3 & 0 & 0 & 1 	
	}
\right] \rightarrow
\left[
	\matrix{
		1 & -2 & 1 & 1 & 0 & 0 \cr
		0 & 1 & -1 & -4 & 1 & 0 \cr	
		0 & 2 & -1 & 2 & 0 & 1
	}
\right] \rightarrow
\left[
	\matrix{
		1 & 0 & 0 & 3 & 0 & 1 \cr
		0 & -1 & 0 & -6 & 1 & -1 \cr
		0& 2 & -1 & 2 & 0 & 1	
	}
\right] \rightarrow
$$
$$
\left[
	\matrix{
		1 & 0 & 0 & 3 & 0 & 1 \cr
		0 & 1 & 0 & 6 & -1 & 1 \cr
		0 & 0 & 1 & 10 & -2 & 1	
	}
\right]
$$

$$
\left[
	\matrix{
		1 & -2 & 1 \cr
		4 & -7 & 3 \cr
		-2 & 6 & -3	
	}
\right]
\left[
	\matrix{
		3 & 0 & 1 \cr
		6 & -1 & 1 \cr
		10 & -2 & 1	
	}
\right] =
\left[
	\matrix{
		1 & 0 & 0 \cr
		0 & 1 & 0 \cr
		0 & 0 & 1	
	}
\right]
$$







\medskip
\noindent{\bf 7.}
(a) We can augment the matrix with both $e_{1}$ and $e_{2} \in \IR^2$ to find the columns of the matrix $B$.
$$
\left[
	\matrix{
		1 & 1 & 1 & 0 & 1 \cr
		0 & 1 & 1 & 1 & 0	
	}
\right]
$$
and
$$
\left[
	\matrix{
		1 & 1 & 1 & 0 & 0 \cr
		0 & 1 & 1 & 1 & 1	
	}
\right]
$$
In both cases we have two free variables, $x_3$ and $x_4$, which we can set equal to $s$ and $t$, respectively. If we set $s = 0$ and $t = 1$, then we have $b_1 = \left[\matrix{ 1 \cr 0 \cr 0 \cr 0 }\right]$ and $b_2 = \left[\matrix{ 0 \cr 0 \cr 0 \cr 1 }\right]$. and so $B = \left[\matrix{ 1 & 0 \cr 0 & 0 \cr 0 & 0 \cr 0 & 1 }\right]$. Indeed
$$
\left[
	\matrix{
		1 & 1 & 1 & 0 \cr
		0 & 1 & 1 & 1	
	}
\right]
\left[
	\matrix{
		1 & 0 \cr
		0 & 0 \cr
		0 & 0 \cr 
		0 & 1 }
\right] = 
\left[
	\matrix{
		1 & 0 \cr
		0 & 1
	}
\right]
$$
(b) Assume there is a $4 x 2$ matrix $C$ such that $CA = I_4$. In order for $I_{1,1} = 1$, $C_{1,1}$ must equal $1$. Then in order for $I_{1,2} = 0$, $C_{1,2}$ must equal $-1$. So we have the first row of Matrix $C$ being $\left[\matrix{ 1 & -1 }\right]$. However we find that if this is the case, then $I_{1,3} = -1$ and this is not the identity matrix. So there is no such matrix $C$.





\medskip
\noindent{\bf 8.}
Assume $A$ has linearly dependent columns. That means there exists some vector $x$, where $x \neq 0$, such that $Ax = 0$. This entails that $A^2$ is also linearly dependent because $A^2 = AA$, and we know there is some $x$ such that $Ax = 0$, so we have $A^2x = A(Ax)$ which becomes $A^2 = A0$ and finally $A^2x = 0$. However, we are given that $A^2$ spans $\IR^n$, which, by the IVT, entails that its columns are linearly independent. Therefore, we have a contradiction and $A$ must have linearly independent columns. 



\medskip
\noindent{\bf 9.} (10 points)

(a)
$$
A =
\left[
	\matrix{
		2 & -8 \cr
		-2 & 7	
	}
\right]
$$
$$
A^{-1} =
det A \left[
	\matrix{
		7 &	8 \cr
		2 & 2
	}
\right] = 
\left[
	\matrix{
		-\frac{7}{2}	& -4 \cr
		-1 & -1
	}
\right]
$$

(b)
$$R(x_1, x_2) = (-\frac{7}{2}x_1 - x_2, -4x_1 - x_2)$$

(c)
$$R \circ T(x_1, x_2)$$
$$= R(2x_1 - 8x_2, -2x_1 + 7x_2) $$
$$
=
(
	-\frac{7}{2}(2x_1 - 8x_2) - (-2x_1 + 7x_2),
	-4(2x_1 - 8x_2) - (-2x_1 + 7x_2)
)
$$
$$= (x_1, x_2)$$

(d)
$$T \circ R(x_1, x_2)$$
$$= T(-\frac{7}{2}x_1 - x_2, -4x_1 - x_2)$$
$$
=
(
	2(-\frac{7}{2}x_1 - x_2) - 8(-4x_1 - x_2),
	-2(-\frac{7}{2}x_1 - x_2) + 7(-4x_1 - x_2)
)
$$
$$= (x_1, x_2)$$
\end{document}

