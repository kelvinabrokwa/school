\documentclass[11pt]{article} % 

\usepackage{amsfonts,amstext}
\usepackage{amsthm}

\setlength{\oddsidemargin}{-0.15in}
\setlength{\topmargin}{-0.5in}
\setlength{\textwidth}{6.5in}
\setlength{\textheight}{9in}

\newcommand{\cP}{{\cal P}}
\newcommand{\IN}{{\bf N}}
\newcommand{\IZ}{{\bf Z}}
\newcommand{\IR}{{\bf R}}
\newcommand{\IQ}{{\bf Q}}
\newcommand{\points}[1]{{\it (#1 Points)}}


\begin{document} \openup .5\jot
%\maketitle

\noindent
{\Large Math 211 \hfill Homework 12 \quad Sample Texfile}

\medskip
\noindent{\bf 1.}
$$x \cdot w = (6)(3) + (-2)(-1) + (3)(-3) = 5$$
$$x \cdot x = 6^2 + 2^2 + 3^2 = 49$$

So

$$
\left(\frac{{\bf x\cdot w}}{{\bf x\cdot x}}\right)
{\bf x} = \left[
	\matrix{
		\frac{30}{49} \cr \frac{-10}{49} \cr \frac{15}{49}	
	}
\right]
$$





\medskip
\noindent{\bf 2.}
$$||u|| = \sqrt{6^2 + 4^2 + 3^2} = \sqrt{61}$$
So the unit vector is:

$$
\left[
	\matrix{
		\frac{-6}{\sqrt{61}} \cr
		\frac{4}{\sqrt{61}} \cr
		\frac{-3}{\sqrt{61}}
	}
\right]
$$





\medskip
\noindent{\bf 3.} 
$$u \cdot v = (12)(2) + (3)(-3) | (-5)(3) = 0$$.

So $u$ and $v$ are orthogonal.




\medskip
\noindent{\bf 4.}
(a)
$$\left[\matrix{2\cr -5\cr -3}\right] \cdot \left[\matrix{4\cr -2\cr 6}\right] = 8 + 10 -18 = 0$$

Since the only remaining vector is $0$ its dot product with the other vectors is also $0$, so the set is orthogonal.

(b) Find a nonzero vector orthogonal to the span of the set.

A vector orthogonal to the span of the set can be found by applying the cross product:

$$
\left[\matrix{2\cr -5\cr -3}\right] \times \left[\matrix{4\cr -2\cr 6}\right] = \left[ \matrix{ -9 \cr -6 \cr 4 } \right]
$$





\medskip
\noindent{\bf 5.} We know the set is orthogonal because we can reduce the matrix whos columns are composed of the set to row echelon form to get 3 pivoting columns (so the set is linearly independent):

$$
\left[
	\matrix{
		3 & 2 & 1 \cr
		-3 & 2 & 1 \cr
		0 & -1 & 4	
	}
\right] \rightarrow \left[
	\matrix{
		3 & 2 & 1 \cr
		0 & 2 & 1 \cr
		0 & 0 & 5
	}
\right]
$$

We can express ${\bf x}=\left[\matrix{5\cr -3\cr 1}\right]$ as a linear combination of set by simly augementing the matrix and solving:

$$
\left[
	\matrix{
		3 & 2 & 1 & 5\cr
		-3 & 2 & 1 & -3\cr
		0 & -1 & 4 & 1	
	}
\right] \rightarrow \left[
	\matrix{
		3 & 0 & 0 & 4 \cr
		0 & 2 & 1 & 1\cr
		0 & 0 & 5 & 3
	}
\right]
$$

So ${\bf x} = \frac{4}{3}{\bf u}_1 + \frac{1}{3}{\bf u}_2 + \frac{1}{3}{\bf u}_3$






\medskip
\noindent{\bf 6.}
Let  ${\bf y}=\left[\matrix{2\cr 6}\right]$ and ${\bf u}=
\left[\matrix{7\cr 1}\right]$. Express ${\bf y} = a{\bf u} + b{\bf z}$
such that ${\bf z}$ is orthogonal to ${\bf u}$.






\medskip
\noindent{\bf 7.} 
Let  ${\bf y} = \left[\matrix{-3\cr 9}\right]$ 
and ${\bf u}=
\left[\matrix{1\cr 2}\right]$.
Find the distance from ${\bf y}$ to the line passing through ${\bf u}$.







\medskip\noindent
{\bf 8.} Let  ${\bf x}_1 = \left[\matrix{-2/3\cr 1/3\cr 2/3\cr}\right]$ 
and ${\bf x}_2=
\left[\matrix{1/3\cr h \cr 0\cr}\right]$.

Determine $h$ so that the two vectors are orthogonal; 

then
find positive numbers $a, b$ so that 
$a{\bf x}_1$ and $b{\bf x}_2$ are orthonormal.


\end{document}
