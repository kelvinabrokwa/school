\documentclass[11pt]{article} % 

\usepackage{amsfonts,amstext}

\setlength{\oddsidemargin}{-0.15in}
\setlength{\topmargin}{-0.5in}
\setlength{\textwidth}{6.5in}
\setlength{\textheight}{9in}

\newcommand{\cP}{{\cal P}}
\newcommand{\IN}{{\bf N}}
\newcommand{\IZ}{{\bf Z}}
\newcommand{\IR}{{\bf R}}
\newcommand{\IQ}{{\bf Q}}
\newcommand{\points}[1]{{\it (#1 Points)}}
\newcommand{\qed}{{\hfill {\rm QED}}}

\def\bu{{\bf u}}
\def\bv{{\bf v}}
\def\bx{{\bf x}}
\def\by{{\bf y}}
\def\be{{\bf e}}
\def\IR{{\mathbb{R}}}

\begin{document}
\openup.5\jot
%\maketitle

\noindent
{\Large Math 211 \hfill Homework 3 \hfill Kelvin Abrokwa-Johnson}


\medskip
\noindent{\bf 1.}
Determine whether the following three vectors are linearly independent.
{\footnotesize
$$\left[\matrix{0\cr 2\cr 3}\right], \  \left[\matrix{0\cr 0\cr -8}\right], \ 
\left[\matrix{-1\cr 3\cr 1}\right].$$} \\
$$
\left[
	\matrix{
		0 & 0 & -1 & 0 \cr
		2 & 0 & 3 & 0 \cr
		3 & -8 & 1 & 0
	}
\right] \rightarrow
\left[
	\matrix{
		0 & 0 & -1 & 0 \cr
		2 & 0 & 0 & 0 \cr
		3 & -8 & 0 & 0 \cr
	}
\right] \rightarrow
\left[
	\matrix{
		0 & 0 & 1 & 0 \cr
		1 & 0 & 0 & 0 \cr
		0 & 1 & 0 & 0
	}
\right]
$$





\medskip
\noindent{\bf 2.}
Determine the value(s) $h$ such that the following three vectors are linearly independent.
$$
{\bf v_1} = {\footnotesize\left[\matrix{3\cr -6\cr 1\cr}\right]}, \
{\bf v_2} = {\footnotesize\left[\matrix{-6\cr 4\cr -3\cr}\right]}, \ 
{\bf v_3} = {\footnotesize\left[\matrix{9\cr h\cr 3\cr}\right]}.
$$
Solution:

$$
\left[
	\matrix{	
		3 & -6 & 9 & 0 \cr
		-6 & 4 & h & 0 \cr
		1 & 3 & 3 & 0
	}
\right] \rightarrow
\left[
	\matrix{
		3 & -6 & 9 & 0 \cr
		0 & -8 & 18+h & 0 \cr
		0 & 1 & 0 & 0 \cr
	}
\right] \rightarrow
\left[
	\matrix{
		3 & -6 & 9 & 0 \cr
		0 & 0 & 18+h & 0 \cr
		0 & 1 & 0 & 0 \cr
	}
\right]
$$
We can see that the vectors are linearly independent when $h \neq -18$. Which would render the system inconsistent, thereby making the trivial case the only solution to the system.




\medskip
\noindent{\bf 3.} Determine (with explanation) the reduced echelon form of a 
$3 \times 3$ matrix $A$ if it has linearly independent 
columns. 





\medskip
\noindent{\bf 4.}
Let 
{\footnotesize 
$$A = \left[\matrix{1 & 2 & 3\cr 0 & 4 & 5 \cr
0 & 0 & 6}\right],  {\bf u} = \left[\matrix{3\cr 6\cr -9\cr} \right], 
{\bf v} = \left[\matrix{a\cr b\cr c\cr}\right].$$} 
Determine $T(\bu)$ and $T(\bv)$ if 
$T: \IR^3 \rightarrow \IR^3$ is the linear transformation defined by $T(\bx) = A\bx$. \\
Solution:
By performing matrix multiplication, we get the following:
$$
T(u) = 
\left[
	\matrix{
		1 & 2 & 3 \cr
		0 & 4 & 5 \cr
		0 & 0 & 6
	}
\right]
\left[
	\matrix{
		3 \cr
		6 \cr
		-9	
	}
\right] = 
\left[
	\matrix{
		12 \cr
		-21 \cr
		-54	
	}
\right]
$$
$$
T(v) = 
\left[
	\matrix{
		1 & 2 & 3 \cr
		0 & 4 & 5 \cr
		0 & 0 & 6
	}
\right]
\left[
	\matrix{
		a \cr
		b \cr
		c	
	}
\right] = 
\left[
	\matrix{
		1 + 2b + 3c \cr
		4b + 5c \cr
		6c	
	}
\right]
$$





\medskip
\noindent{\bf 5.}
Let
{\footnotesize
$$A = \left[\matrix{1&-2&3\cr 0&1&-3\cr 2&-5&6}\right], \quad {\bf b} = 
\left[\matrix{-6\cr -4\cr -5\cr}\right].$$}
Find a vector ${\bf x}$ such that $A{\bf x}={\bf b}$. \\
Solution:
$$
\left[
	\matrix{
		1 & -2 & 3 & -6 \cr
		0 & 1 & -3 & -4 \cr
		2 & -5 & 6 & -5
	}
\right] \rightarrow
\left[
	\matrix{
		1 & -2 & 3 & -6 \cr
		0 & 1 & -3 & -4 \cr
		0 & -1 & 0 & 7 
	}
\right] \rightarrow
\left[
	\matrix{
		1 & -2 & 3 & 6 \cr
		0 & 0 & 1 & -1 \cr
		0 & 1 & 0 & -7
	}
\right]
$$
And so from that row reduction we can see that $x_1 = -5$, $x_2 = -7$, and $x_3 = -1$. So,
$$
x = 
\left[
	\matrix{
		-5 \cr
		-1 \cr
		-7	
	}
\right]
$$


\medskip
\noindent{\bf 6.}
Let  
{\footnotesize
$$A=\left[\matrix{3&2&10&-6\cr 1&0&2&-4\cr 0&1&2&3\cr 1&4&10&8\cr}\right].$$}
Find all vectors that are mapped into the zero vector
by the transformation ${\bf x} \rightarrow A{\bf x}$.  \\
Solution:
$$
\left[
	\matrix{
		3 & 2 & 10 & -6 & 0 \cr
		1 & 0 & 2 & -4 & 0 \cr
		0 & 1 & 2 & 3 & 0 \cr
		1 & 4 & 10 & 8 & 0
	}
\right] \rightarrow
\left[
	\matrix{
		3 & 2 & 10 & -6 & 0 \cr
		0 & 2 & 4 & 6 & 0 \cr
		0 & 1 & 2 & 3 & 0 \cr
		0 & -10   & -20 & -30 & 0
	}
\right] \rightarrow
\left[
	\matrix{
		3 & 2 & 10 & -6 & 0 \cr
		0 & 1 & 2 & 3 & 0 \cr
		0 & 1 & 2 & 3 & 0 \cr
		0 & 1 & 2 & 3 & 0
	}
\right] \rightarrow
\left[
	\matrix{
		3 & 2 & 10 & -6 & 0 \cr
		0 & 1 & 2 & 3 \cr
		0 & 0 & 0 & 0 & 0 \cr
		0 & 0 & 0 & 0 & 0
	}
\right]
$$









\medskip
\noindent{\bf 7.}
Let
{\footnotesize
$${\bf x}=\left[\matrix{x_1\cr x_2}\right], 
{\bf v_1}=\left[\matrix{-3\cr 5}\right],
{\bf v_2}=\left[\matrix{7\cr -2}\right],$$}
and let $T: \IR^2 \rightarrow \IR^2$ 
such that $T({\bf x}) = x_1{\bf v_1}+x_2{\bf v_2}$. 
Find the matrix $A$ such that $T({\bf x}) = A{\bf x}$ for each 
${\bf x} \in \IR^2$. \\
Solution:
$$
\left[
	\matrix{
		-3 & 7 \cr
		5 & -2	
	}
\right]
$$



\medskip
\noindent{\bf 8.}
Find the standard matrix of a transformation $T$ that 
reflects points through the horizontal $x_1$ axis and then reflects the 
point through the line $x_2=x_1$. \\
Solution:
$$
T(e_1) = 
T(\left[
	\matrix{ 1 \cr 0 }
\right]) =
\left[
	\matrix{ 0 \cr 1 }
\right],
T(e_2) = 
T(\left[
	\matrix{ 0 \cr 1 }
\right]) =
\left[
	\matrix{ -1 \cr 0 }
\right]
$$
And so the solution is:
$$
\left[
	\matrix{
		 0 & -1 \cr
		 1 & 0
	}
\right]
$$



\medskip
\noindent{\bf 9.}
Suppose $T(x_1, x_2)=(x_1+4x_2, 0,x_1-3x_2,x_1).$ 
Find a matrix $A$ such that
$$A\left[\matrix{x_1\cr x_2\cr}\right] = 
{\footnotesize
\left[\matrix{x_1+4x_2\cr0\cr x_1-3x_3\cr x_1\cr}\right]}.$$  \\
Solution:
We can rewrite like so:
$$
T(x_1, x_2) = 
\left[
	\matrix{
		x_1 + 4x_2 \cr
		0 \cr
		x_1 - 3x_2 \cr
		x_1	
	}
\right]
$$
And so we see that:
$$
A =
\left[
	\matrix{
	1 & 4 \cr
	0 & 0 \cr
	1 & -3 \cr
	1 & 0	
	}
\right]
$$ 


\end{document}