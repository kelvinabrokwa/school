\documentclass[11pt]{article} % 

\usepackage{amsfonts,amstext}
\usepackage{amsthm}

\setlength{\oddsidemargin}{-0.15in}
\setlength{\topmargin}{-0.5in}
\setlength{\textwidth}{6.5in}
\setlength{\textheight}{9in}

\newcommand{\cP}{{\cal P}}
\newcommand{\IN}{{\bf N}}
\newcommand{\IZ}{{\bf Z}}
\newcommand{\IR}{{\bf R}}
\newcommand{\IQ}{{\bf Q}}
\newcommand{\points}[1]{{\it (#1 Points)}}


\begin{document} 
%\maketitle

\noindent
{\large Math 211 \qquad  Homework 11 \hfill Kelvin Abrokwa-Johnson}

 
\medskip
\noindent{\bf 1.}
When $\lambda = 2$ our solution set to solving the system set equal to $0$ is $Span \left\{ \left[ \matrix{ -1 \cr 1 \cr 0 } \right] \right\}$ and when $\lambda = 3$ we have $Span \left\{ \left[ \matrix{ -2 \cr 0 \cr 1 } \right], \left[ \matrix{ 0 \cr 1 \cr 0 } \right] \right\}$. So we have:

$$
P = \left[
	\matrix{
	-1 & -2 & 0	\cr
	1 & 0 & 1 \cr
	0 & 1 & 0
	}
\right],
D = \left[
	\matrix{
		2 & 0 & 0 \cr
		0 & 3 & 0 \cr
		0 & 0 & 3	
	}
\right]
$$






\medskip\noindent
{\bf 2.}
(a) We know that the dimension of Nul$(A - \lambda_3 I) = 2$ because, by Theorem 6, an $nxn$ matrix with $n$ distinct eigenvalues is diagonalizable. Since, $A - \lambda_1 I$ and $A - \lambda_2 I$ produce 5 eigenvalues, $A - \lambda_3 I$ must produce 2.

(b) If $A$ is not diagonalizable then dim Nul$(A - \lambda_3I) = 1$.
(c)




\medskip
\noindent{\bf 3.}
We see immediately that $[T({\bf d}_1)] = \left[ \matrix{ 3 \cr -3 } \right]$ and $[T({\bf d}_2)] = \left[ \matrix{ -2 \cr 5 } \right]$. So the matrix for $T$ relative to $\mathcal{D}$ and $\mathcal{B}$ is $\left[ \matrix{ 3 & -2 \cr -3 & 5 } \right]$.




\medskip
\noindent{\bf 4.}
(a)
$$
T(3 - 2t + t^2) = (3 - 2t + t^2) + 2t^2(3 - 2y + t^2) = 3 - 2t + 7t^2 -4t^3 + 2t^4
$$

(b) First we show that the transformation is closed under addition. Take $p(t) = a_1 + b_1t + c_1t^2$ and $q(t) = a_2 + b_2t + c_2t^2$
$$
T(p(t) + q(t)) = T((a_1 + a_2) + (b_1 + b_2)t + (c_1 + c_2)t^2)
$$
$$
= ((a_1 + a_2) + (b_1 + b_2)t + (c_1 + c_2)t^2) + 2t^2((a_1 + a_2) + (b_1 + b_2)t + (c_1 + c_2)t^2)
$$
$$
= (a_1 + b_1t + c_1t^2 + 2t^2a_1 + 2t^3b_1 + 2t^4c_1) + (a_2 + b_2t + c_2t^2 + 2t^2a_2 + 2t^3b_2 + 2t^4c_2)
$$
$$
= T(p(t)) + T(q(t))
$$
Now we show that it is closed under scalar multiplication. Take $p(t) = a + bt + ct^2$ and some scalar $k$:
$$T(kp(t)) = T(ka + kbt + kct^2)$$
$$= ka + kbt + kct^2 + 2kat^2 + 2kbt^3 + 2kct^4$$
$$= k(a + bt + ct^2 + 2at^2 + 2bt^3 + 2ct^4)$$
$$= kT(p(t))$$

(c) We simply define the transformation on the standard polynomials to get:
$$T(1) = 1 + 2t^2$$
$$T(t) = t + 2t^3$$
$$T(t^2) = t^2 + 2t^3$$

so the matrix $T$ relative to the bases is:
$$
M = \left[
	\matrix{
		1 & 0 & 0 \cr
		0 & 1 & 0 \cr
		2 & 0 & 1 \cr
		0 & 2 & 0 \cr
		0 & 0 & 2	
	}
\right]
$$






\medskip
\noindent{\bf 5.}
$$
T(4b_1 - 3b_2) =
\left[
	\matrix{
		0 & 0 & 1 \cr
		2 & 1 & -2 \cr
		1 & 3 & 1	
	}
\right]
\left[
	\matrix{ 4 \cr -3 \cr 0 }
\right]
=
\left[
	\matrix{ 0 \cr 5 \cr -5}
\right]
$$

Which is
$$
5b_2 - 5b_3
$$



\medskip
\noindent{\bf 6.}
In essence, we must find a basis $\mathcal{B}$ such that the transformation $T$ in that basis is a diagonal matrix: $T([x]_{\mathcal{B}}) = D[x]_{\mathcal{B}}$ where $D$ is a diagonal matrix.  The $\mathcal{B}$ coordinates of a vector $x$, $[x]_{\mathcal{B}}$, can be transformed if we first convert it back to the standard basis (by multiplying by the change of basis matrix $P_{\mathcal{C} \leftarrow \mathcal{B}}$), multiply by $A$ and then convert back to the $\mathcal{B}$ basis (by multiplying it by $P_{\mathcal{B} \leftarrow \mathcal{C}}$). So $T([x]_{\mathcal{B}}) = D[x]_{\mathcal{B}}$ becomes $T([x]_{\mathcal{B}}) =  P_{\mathcal{B} \leftarrow \mathcal{C}} A P_{\mathcal{C} \leftarrow \mathcal{B}} = D[x]_{\mathcal{B}}$.

Now it is clear that we are looking for a diagonal matrix similar to $A$. We know that the diagonal matrix $D$ similar to $A$ is composed of the eigenvalues. So we have $A = PDP^{-1}$ where the columns of $P$ are the eigenvectors of $A$. So $D = \left[ \matrix{ 5 & 0 \cr 0 & -1 } \right] = [T]_{\mathcal{B}}$ and $P = \left[ \matrix{ 1 & -1 \cr 1 & 1 } \right]$ and the basis $\mathcal{B} = \left\{ \left[ \matrix{ 1 \cr 1 } \right], \left[ \matrix{ -1 \cr 1 } \right] \right\}$.




\medskip\noindent
\medskip
\noindent{\bf 7.}
Similar to the problem above, we can rephrase the problem as finding a diagonal vector such that the matrix of transformation $\left[ \matrix{ 2 & 3 \cr 3 & 2 } \right]$. This matrix has eigenvalues $\lambda = 5,-1$ so the eigenvectors are $Span \left\{ \left[ \matrix{ 1 \cr 1 } \right]\right\}$ and $Span \left\{ \left[ \matrix{ -1 \cr 1 } \right]\right\}$. So a basis for said polynomial space is $\mathcal{B} = \{1 + t, -1 + t\}$.


\end{document}