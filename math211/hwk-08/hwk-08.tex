\documentclass[11pt]{article} % 

\usepackage{amsfonts,amstext}
\usepackage{amsthm}

\setlength{\oddsidemargin}{-0.15in}
\setlength{\topmargin}{-0.5in}
\setlength{\textwidth}{6.5in}
\setlength{\textheight}{9in}

\newcommand{\cP}{{\cal P}}
\newcommand{\IN}{{\bf N}}
\newcommand{\IZ}{{\bf Z}}
\newcommand{\IR}{{\bf R}}
\newcommand{\IQ}{{\bf Q}}
\newcommand{\points}[1]{{\it (#1 Points)}}


\begin{document} 
%\maketitle

\noindent
{\Large Math 211 \quad Homework 8
\hfill Kelvin Abrokwa-Johnson}

\medskip
\noindent{\bf 1.}
We see that $x_4$ and $x_2$ are free variables. Using this we can set them to $t$ and $s$ and solve for the null space:

$$
Nul A = t \left[
	\matrix{ 0 \cr 0 \cr 0 \cr 1 }
\right] + s \left[
	\matrix{ 3 \cr 1 \cr 0 \cr 0 }
\right]
$$

\medskip
\noindent{\bf 2.}
\begin{enumerate}
	\item Yes, $Ax  = w$, for $x = \left[ \matrix{ 0 \cr 0 \cr 0 \cr -1 } \right]$.
	
	\item Yes, $Aw = \left[\matrix{10&-8&-2&-2\cr 0&2&2&-2\cr 1&-1&6&0\cr 1&1&0&-2}\right] \left[\matrix{2\cr 2\cr 0\cr 2}\right] = \left[ \matrix{ 0 \cr 0 \cr 0 \cr 0 } \right]$.
\end{enumerate}

\medskip
\noindent{\bf 3.}
In order to find the set of all functions  for which $T(p) = \left[ \matrix{ p(0) \cr p(0) } \right]$ we must simply ensure that the constant term of the polynomial is $0$ (because all other terms will be multiplied by $0$). So, the kernel can be defined as $Span\{ t, t^2 \}$.

The range given that the generic form of the polynomial is $p(t) = c_1 + c_2t + c_3t^2$ is $\left\{ \left[\matrix{ c_1 \cr c_1 }\right] : c_1 \in R \right\}$ because after the transformation all terms except for the constant will become $0$.

\medskip
\noindent{\bf 4.} 
We can find both the Nul$A$ and Col$A$ by row reducing like so:

$$
A =
\left[
	\matrix{
		-2 & 4 & -2 & -4\cr 
		 2 & -6 & -3 & 1 \cr
		-3 & 8 & 2 & -3
	}
\right] \rightarrow
\left[
	\matrix{
		-2 & 4 & -2 & -4\cr 
		 0 & -2 & -5 & -3 \cr
		0 & 0 & 0 & 0
	}
\right]
$$
$$
Nul A = \left\{
	\left[
		\matrix{ -11 \cr -5 \cr 1 \cr 0 }
	\right],
	\left[
		\matrix{ 8 \cr -3 \cr 0 \cr 1 }
	\right]
\right\},
Basis = \left\{
	\left[
		\matrix{ -2 \cr 2 \cr -3 }
	\right],
	\left[
		\matrix{ 4 \cr -6 \cr 8 }
	\right]
\right\}
$$


\medskip
\noindent{\bf 5.}
We can find the basis by simply row reducing and choosing columns with leading non-zeros.

$$
\left[
	\matrix{
		1 & -2 & 3 & 5 & 2 \cr
		0 & 0 & -1 & -3 & -1 \cr
		0 & 0 & 1 & 3 & 1 \cr
		1 & 2 & -1 & -4 & 0
	}
\right] \rightarrow
\left[
	\matrix{
		1 & -2 & 3 & 5 & 2 \cr
		0 & 0 & -1 & -3 & -1 \cr
		0 & 0 & 0 & 0 & 0 \cr
		0 & 4 & -4 & -9 & -2
	}
\right] \rightarrow
\left[
	\matrix{
		1 & -2 & 3 & 5 & 2 \cr
		0 & 0 & -1 & -3 & -1 \cr
		0 & 0 & 0 & 0 & 0 \cr
		0 & 0 & 0 & -5 & -2
	}
\right]
$$

$$
basis = \left\{
	\left[
		\matrix{ 1 \cr 0 \cr 0 \cr 1 }
	\right],
		\left[
		\matrix{ 3 \cr -1 \cr 1 \cr -1 }
	\right],
		\left[
		\matrix{ 5 \cr -3 \cr 3 \cr -4 }
	\right]
\right\}
$$


\medskip
\noindent{\bf 6.}
Since $2v_1 - v_2 -v_3 = 0$ implied that $v_3$ can be expressed as a multiple of $v_1$ and $v_2$ because $2v_1 - v_2 = v_3$, we can eliminate it from our basis. It is clear that $v_1$ and $v_2$ are not multiples of one another so the basis $ = \{ v_1, v_2 \}$

\medskip\noindent{\bf 7.} Find a basis for the vector space $H$ of 
continuous functions spanned by the set 
$$\{\cos t, \sin t, \sin 2t, \sin t\cos t\}.$$
 
It is clear that sin $\sin t$ can not be expressed as a multiple of $\cos t$, so those two are in the basis. We know, howerver, from trigonometric identities, that $\frac{1}{2}\sin 2t = \sin t \cos t$. Since $\sin t \cos t$ can be expressed as a multiple of the other functions we can disclude it from the basis.
 
So the basis $ = \{ \cos t, \sin t, \sin 2t \}$

\end{document}
