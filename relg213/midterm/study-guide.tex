\documentclass[11pt]{article} % 

\setlength{\oddsidemargin}{-0.15in}
\setlength{\topmargin}{-0.5in}
\setlength{\textwidth}{6.5in}
\setlength{\textheight}{9in}
\setlength\parindent{0pt}

\begin{document}

\begin{center}
{\bf Hindiusm Midterm Study Guide}
\end{center}

\begin{itemize}

\item
{\bf tirtha}: ford or crossing; places for connection between humans and the divine

\item
{\bf Yuga}
\begin{itemize}
\item {\bf Krita/Satya}: 4 legs
\item {\bf Treta}: 3 legs
\item {\bf Dvapar}: 2 legs
\item {\bf Kali}: 1 leg
\end{itemize}

\item
{\bf Manu and the Fish}: fish (first Vishnu avatara) is in danger, Manu saves it, fish helps him survive flood

\item
{\bf The Aryans}: nomadic people who migrated from the caucauses; linguistic group

\item
{\bf The Veda (Rig Veda)}: oldest Hindu texts

\item
{\bf Indra}: king of gods; killed demon and allowed creation of world; realizes insignificance by seeing parade of ants that were once gods

\item
{\bf Agni}: fire god; intermediary between humans and gods

\item
{\bf Soma}: gave gods strength to fight demons; moon is filled with it; also a god

\item
{\bf Yajna}: the sacrifice; ritual of fire sacrifice

\item
{\bf The Upanishads}: "sitting down near to"; secret knowledge; big questions; more inward approach that Vedas; ritual action becomes less important than knowledge

\item
{\bf Tapas}: creative, transformative heat

\item
{\bf Vac}: cosmic feminine sound energy; also god; Om

\item
{\bf Brahman}: priest

\item
{\bf Atman}: the inner, universal self that trancends the ego

\item
{\bf Tat tvam asi}: that is how you are; Svetaketu

\item
{\bf Jnana}: knowledge of ritual practice

\item
{\bf Samasara}: cycle of rebirth

\item
{\bf Karma}: a universal force that one accumulates based on deeds

\item
{\bf Moksha}: enlightenment, escape from samsara (rebirth)

\item
{\bf Varna}: social classes
\begin{itemize}
\item {\bf Brahman}: priests
\item {\bf Kshatriya}: warriors
\item {\bf Vaishya}: working class, merchants, etc.
\item {\bf Shudra}: servants
\end{itemize}

\item
{\bf Ashrama}
\begin{itemize}
\item 4 ashramas
\end{itemize}

\item
{\bf Jati}: subclasses of varnas

\item
{\bf Dharma}: that which upholds and sustains; truth duty, righteousness law; cosmic principle; human behavior

\item
{\bf Shruti}: that which is heard; more important; Vedas, Upanishads, Baghavad Gita, Tantras (higher than vedas)

\item
{\bf Smriti}: that which is remembered; Mahabharata, Ramayana, Puranas

\item
{\bf Three (or Four) Aims of Life}
\begin{itemize}
\item dharma
\item artha: money, wealth, political power, success
\item kama: pleasure and desire
\item moksha (later)
\end{itemize}

\item
{\bf Kama}: desire

\item
{\bf Artha}: an aim of life; Upanishadic; money, wealth, political power, success

\item
{\bf Mahabharata}: story of war between Pandavas and Kauravas

\item
{\bf Pandavas and Kauravas}
\begin{itemize}
\item Pandavas: 5 sons of Pandu
\item Kauravas: 100 sons of Dhritarashtra
\end{itemize}

\item
{\bf Bhagavad Gita}: portion of Mahabharata where Krishna councels Arjuna

\item
{\bf Arjuna}: protagonish of Bhagavad Gita; discourses with Vishnu his charioteer

\item
{\bf Three Yogas (of Gita)}: yoga is discipline
\begin{itemize}
\item Bhakti: devotion
\item Karma: action
\item Jnana: knowledge
\end{itemize}

\item
{\bf Ramayana}: story of Rama

\item
{\bf Valmiki}: writer of oldest full version of Ramayana

\item
{\bf Rama}: avatara of Vishnu; protagonist of Ramayana; defeats Ravana

\item
{\bf Sita}: wife of Rama; born of the earth; daughter of Janaka

\item
{\bf Lakshmana}: half brother of Rama; goes with him into the forest when he is banished

\item
{\bf Hanuman}: monkey servant of Rama

\item
{\bf Ravana}: demon; gained power through worship of Shiva who granted him boon that he can't be killed by gods or demons

\item
{\bf Tulsida/Ramcaritmanas}: Author of North Indian version of Ramayana; Rama is more god-like

\item
{\bf Ram Lila}: festival in which Ramayana was reenacted

\item
{\bf The Puranas}: stories and myths

\item
{\bf The Tantras}: liturgical guidelines

\item
{\bf Brahmanization \& Sanskritization}
\begin{itemize}
\item Brahmanization: top down
\item Sanskritization: bottom up
\end{itemize}

\item
{\bf Nirguna v. saguna}: representation of god
\begin{itemize}
\item gun: quality
\item nir: without
\item sa: with
\end{itemize}

\item
{\bf Vishnu}: sustainer; many avataras

\item
{\bf avatara}: descent; human incarnation of god

\item
{\bf Shiva}: portrayed about ascetic

\item
{\bf Devi (Mahadevi)}: the goddess

\end{itemize}

\begin{center}
{\bf Essay Questions}
\end{center}



\begin{enumerate}

\item
\begin{itemize}
	\item agni
	\begin{itemize}
		\item fire (smoke)
		\item generate tapas
	\end{itemize}

	\item sacred sound
	\begin{itemize}
		\item power in sound words
		\item vac: cosmic feminine sound energy; the divine as sound;
					sometimes equated with Om
		\item everything is different vibrations of one universal voice
		\item mantras
	\end{itemize}

	\item family
	\begin{itemize}
		\item cycle of reproduction
		\item duty to procreate
		\item only married men could sacrifice
		\item man produces semen and sacrifices it into woman and they create child
	\end{itemize}

\end{itemize}


\pagebreak

\item
\begin{itemize}
	\item Upanishad's basic understanding of the world as we know it
	\begin{itemize}
		\item trapped in samsara, bound by earthly desire
		\item escape through moksha
		\item sacrifice becomes less important
		\item emphasis on jnana
		\item emphasis on atman
		\item karma
		\item shruti
	\end{itemize}
\end{itemize}

\item
\begin{itemize}
	\item Bhakti
	\begin{itemize}
		\item devotion
		\item beginning shift toward worship
	\end{itemize}
	\item Karma
	\begin{itemize}
		\item action without desire
		\item based on dharmic duty rather than result of action
	\end{itemize}
		\item Jnana
	\begin{itemize}
		\item knowledge of sacrificial practice
		\item attain knowledge to divorce self from ego
		\item sacrifice fruits of action
	\end{itemize}
\end{itemize}
Gita synthesizes jnana and karma yoga in pursuit of bhakti yoga

\item
\begin{itemize}
	\item Rama
	\begin{itemize}
		\item accomponies sage to protect him
		\item fulfills father's boon
	\end{itemize}
	\item Sita
	\begin{itemize}
		\item goes with husband into forest
		\item accepts ascetic into home
	\end{itemize}
	\item Ramayana vs. Mahabharata
	\begin{itemize}
		\item dharma is more intact in Ramayana yuga
		\item more subtle conflicts in Mahabharata; addresses 						                 contradictory parts of dharma
	\end{itemize}
\end{itemize}

\end{enumerate}

\end{document}