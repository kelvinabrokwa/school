\documentclass[11pt]{article} % 

\usepackage{listings}
\usepackage{amsmath}

\setlength{\oddsidemargin}{-0.15in}
\setlength{\topmargin}{-0.5in}
\setlength{\textwidth}{6.5in}
\setlength{\textheight}{9in}
\setlength\parindent{0pt}


\begin{document} 
%\maketitle

\noindent Kelvin Abrokwa-Johnson \\
14 March 2016 \\
CS 304

\begin{center} Homework 1 \end{center}


{\bf 2.67}

A. The code fails to comply with the C standard because it attempts to shift beyond the word size. In many systems the shift amount will be applied $mod$ the word size so an attempt to shift by $32$ with result in a $32 \mod 32 = 0$ shift.

B. \\
\noindent\rule{12cm}{0.4pt}
\begin{lstlisting}[language=C]
int int_size_is_32() {
	int i = 1 << 31;
	int j = ~i;
	return i > j;
}
\end{lstlisting}
\noindent\rule{12cm}{0.4pt}


C. \\
\noindent\rule{12cm}{0.4pt}
\begin{lstlisting}[language=C]
int int_size_is_32() {}
\end{lstlisting}
\noindent\rule{12cm}{0.4pt}

\vspace{0.5in}
{\bf 2.71}

A. SAY WHY THIS DOESN'T WORK.

B. \\
\noindent\rule{12cm}{0.4pt}
\begin{lstlisting}[language=C]
int xbyte(packed_t word, int bytenum) {
	word = word << ((3 - bytenum) << 3);
	most_sig = (1 << 31) & word; // tells us what will be filled on the next right shift
	word = word >> 24;
	return word - ((most_sig << 31) >> 24);
}
\end{lstlisting}
\noindent\rule{12cm}{0.4pt}

\vspace{0.5in}
{\bf 2.76}

A. $(x << 4) + x$ \\
B. $x - (x << 3)$ \\
C. $(x << 6) - (x << 2)$ \\
D. $(x << 4) - (x << 7)$

\pagebreak
{\bf 2.80}

A. $ \sim 0 << k$

B. 

\vspace{0.5in}
{\bf 2.81}
\end{document}