\documentclass[11pt]{article} % 

\setlength{\oddsidemargin}{-0.15in}
\setlength{\topmargin}{-0.5in}
\setlength{\textwidth}{6.5in}
\setlength{\textheight}{9in}
\setlength\parindent{0pt}

\newcommand{\cP}{{\cal P}}
\newcommand{\IN}{{\bf N}}
\newcommand{\IZ}{{\bf Z}}
\newcommand{\IR}{{\bf R}}
\newcommand{\IQ}{{\bf Q}}
\newcommand{\points}[1]{{\it (#1 Points)}}
\newcommand{\qed}{{\hfill {\rm QED}}}

\begin{document} 

\begin{center}
{\bf Principles of Programming Languages Midterm Study Guide}
\end{center}

{\bf Syntax}
\medskip

\begin{itemize}

\item {\bf syntax}: a precise description of all its grammatically correct programs

\item {\bf lexical syntax}: all the basic symbols of the language

\item {\bf concrete syntax}: rules for writing expressions, statements and programs

\item {\bf abstract syntax}: internal representation of the program, favoring content over form

\item {\bf metalanguage}: a language used to define other languages

\item {\bf grammar}: a metalanguage used to define the syntax of a language

\end{itemize}

{\bf Backus-Naur Form (BNF)}: stylized version of a context-free grammar

\medskip
Set of

\begin{itemize}
\item $productions$: $P$
\item $terminal$ symbols: $T$
\item $nonterminal$ symbols: $N$
\item $start$ symbol: $S \in N$
\end{itemize}

A $production$ has the form $A \rightarrow \omega$ where $A \in N$ and $\omega \in (N \cup T)$

\medskip
{\bf Parse Trees}

A graphical representation of a derivation.

\begin{itemize}
\item each internal node of the tree corresponds to a step in the derivation
\item the children of a node represents a right-hand side of a production
\item each leaf node represents a symbol of the derived string, reading from left to right
\end{itemize}

\medskip
{\bf Associativity and Precedence}
\medskip

A grammar is {\bf ambiguous} if one of its strings has two or more different parse trees.

\medskip
{\bf Extended BNF (EBNF)}
\medskip

{\bf BNF}:
\begin{itemize}
\item recursive for iteration
\item nonterminals for grouping
\end{itemize}

{\bf EBNF: additional metacharacters}:
\begin{itemize}
\item { } for a series of zero of more
\item ( ) for a list, must pick one
\item $[ ]$ for an optional list, pick one or none
\end{itemize}

We can always write an EBNF grammar as a BNF grammar

\medskip
{\bf Identifier}: sequence of letters and digits, starting with a letter

\medskip
{\bf Concrete Syntax}:  based on a parse of its Tokens

\pagebreak
{\bf Lexer}:

\begin{itemize}
\item input: characters
\item output: tokens
\item separate
\begin{itemize}
\item speed: 75\% of time for non-optimizing
\item simpler design
\item character sets
\item end of line convention
\end{itemize}
\end{itemize}

{\bf Parser}:
\begin{itemize}
\item Based on BNF/EBNF grammar
\item Input: tokens
\item Output: abstract syntax tree (parse tree)
\item Abstract syntax: parse tree with punctuation, many nonterminal discarded
\end{itemize}

{\bf Semantic Analysis}
\begin{itemize}
\item Check that all identifiers and declared
\item Perform type checking
\item Insert implied conversion operators (i.e., make them explicit)
\end{itemize}

{\bf Code Optimization}
\begin{itemize}
\item Evaluate constant expressions at compile-time
\item Reorder code to improve cache performance
\item Eliminate common subexpressions
\item Eliminate unnecessary code
\end{itemize}

{\bf Code Generation}
\begin{itemize}
\item Output: machine code
\item Instruction selection
\item Register management
\item Peephole optimization
\end{itemize}

\pagebreak

{\bf Interpreter}
\begin{itemize}
\item Replaces last 2 phases of a compiler
\item Input:
\begin{itemize}
\item Mixed: intermediate code
\item Pure: stream of ASCII characters
\end{itemize}
\item Mixed interpreters
\begin{itemize}
\item Java, Perl, Python, Haskell, Scheme
\end{itemize}
\item Pure interpreters
\begin{itemize}
\item most Basic, shell commands
\end{itemize}
\end{itemize}

{\bf Binding}: an association between an entity (such as a variable) and a property (such as its value)
\begin{itemize}
\item {\bf static}: if the association occurs before run-time
\item {\bf dynamic}: if the association occurs at run-time
\item The {\bf lifetime} of a variable name refers to the time interval during which memory is allocated
\end{itemize}

{\bf Scope}: the collection of statements which can access the name binding
\begin{itemize}
\item {\bf static scoping}: a name is bound to a collection of statements according to its position in source program
\item same as {\bf lexical scoping}
\end{itemize}

{\bf Symbol Table}: a data structure kept by a translator that allows it to keep track of each declared name and its binding

\medskip
{\bf Dynamic Scoping}: in dynamic scoping, a name is bound to its most recent declaration based on the program's call history

\medskip
{\bf Overloading}: uses the number or type of parameters to distinguish among identical function names or operators

\medskip
{\bf Lifetime}: the time interval during which the variable has been allocated a block of memory

\medskip
{\bf type}: a collection of values and operations on those values

\medskip
{\bf type error}: any error that arises because an operation is attempted on a data type for which it is underfined
\begin{itemize}
\item {\bf type system}: provides a basis for detecting type errors
\item {\bf statically typed}: the types of all variables are fixed when they are declared at compile time
\item {\bf dynamically type}: the type of a variable can vary at run time depending on the value assigned
\item {\bg strongly typed}: type system that allows all type errors in a program to be detected at either compile time or at run time
\begin{itemize}
\item can be either statically or dynamically typed
\end{itemize}
\end{itemize}




\end{document}